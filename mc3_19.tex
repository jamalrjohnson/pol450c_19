\documentclass{beamer}

%\usepackage[table]{xcolor}
\mode<presentation> {
  \usetheme{Boadilla}
%  \usetheme{Pittsburgh}
%\usefonttheme[2]{sans}
\renewcommand{\familydefault}{cmss}
%\usepackage{lmodern}
%\usepackage[T1]{fontenc}
%\usepackage{palatino}
%\usepackage{cmbright}
  \setbeamercovered{transparent}
\useinnertheme{rectangles}
}
%\usepackage{normalem}{ulem}
%\usepackage{colortbl, textcomp}
\setbeamercolor{normal text}{fg=black}
\setbeamercolor{structure}{fg= black}
\definecolor{trial}{cmyk}{1,0,0, 0}
\definecolor{trial2}{cmyk}{0.00,0,1, 0}
\definecolor{darkgreen}{rgb}{0,.4, 0.1}
\newtheorem{prop}{Proposition}
\newtheorem{defn}{Definition}
\newtheorem{cor}{Corollary}
\newtheorem{lem}{Lemma}

\usefonttheme{professionalfonts} % using non standard fonts for beamer
\usefonttheme{serif}

\beamertemplatesolidbackgroundcolor{white}  \setbeamercolor{alerted
text}{fg=red}
\usepackage{multirow}

\setbeamertemplate{caption}[numbered]\newcounter{mylastframe}

%\begin{beamerboxesrounded}[upper=uppercol,lower=lowercol,shadow=true]{Block}
%$A = B$.
%\end{beamerboxesrounded}}
\renewcommand{\familydefault}{cmss}
\usepackage[all]{xy}
\usepackage{tikz}
\usetikzlibrary{arrows}
%\usepackage{palatino}
%\usepackage{eulervm}
\usecolortheme{lily}
\usefonttheme{serif}

%\usepackage[latin1]{inputenc}

\title[Methodology III] % (optional, nur bei langen Titeln nötig)
{Political Methodology III: Model Based Inference}

\author{Justin Grimmer}
\institute[Stanford University]{Professor\\Department of Political Science \\  Stanford University}
\vspace{0.3in}

\date{April 9th, 2019}

\begin{document}
\begin{frame}
\titlepage
\end{frame}




\begin{frame}
\frametitle{Statistical Inference}


\begin{itemize}
\item[-] Model based inference:
\item[-] \alert{Assume}: data generated via distributional process
\item[-] Thursday: Derived a likelihood for single/bivariate problems
\item[-] Today: Interested in conditional relationships
\item[-] Today: Build Linear Model Using Likelihood
\end{itemize}
Bivariate \alert{$\leadsto$} Multivariate $\leadsto$ Computational Model
\end{frame}


\begin{frame}
\frametitle{Regression Model$\leadsto$ Time for Change}

Model incumbent vote share, $Y_{i}$
\begin{itemize}
\item[1)] $X_{i1} = $Incumbent Presidential Popularity (Net Approval/Disapproval)
\item[2)] $X_{i2} = $GDP Growth
\item[3)] $X_{i3} = $First-term Incumbent In Race
\end{itemize}
$\boldsymbol{X}_{i} = (X_{i1}, X_{i2}, X_{i3})$





\end{frame}


\begin{frame}
\frametitle{Bivariate Regression Model via Maximum Likelihood}

Linear regression model:
\begin{eqnarray}
Y_{i} & = & \beta_{0} + \beta_{1}X_{i1} + \epsilon_{i} \nonumber \\
\epsilon_{i} & \sim & \text{Normal}(0, \sigma^{2}) \nonumber
\end{eqnarray}
\pause
\invisible<1>{Equivalently:

\begin{eqnarray}
Y_{i} & \sim & \text{Normal}(\mu_{i}, \sigma^{2}) \nonumber \\
E[Y_{i}| X_{i1} ] = \mu_{i} & = & \beta_{0}  + \beta_{1} X_{i1} \nonumber
\end{eqnarray}} \pause

\invisible<1-2>{Parameters:
\begin{eqnarray}
\boldsymbol{\beta} & = & (\beta_{0} , \beta_{1}) \nonumber \\
\sigma^2 && \nonumber
\end{eqnarray}}


%%The key here is get them to recognize the assumptions that they are making
\end{frame}


\begin{frame}
\frametitle{Bivariate Regression Model via Maximum Likelihood}


\begin{eqnarray}
L(\boldsymbol{\beta}, \sigma^2 | \boldsymbol{Y}, \boldsymbol{X}) & = &  f(\boldsymbol{Y} | \boldsymbol{\beta}, \sigma^2,  \boldsymbol{X}) \nonumber \\ \pause
\invisible<1>{&= & \prod_{i=1}^{N} f(Y_{i} | \boldsymbol{\beta}, \sigma^2, X_{i1}) \nonumber \\} \pause
\invisible<1-2>{& = & \prod_{i=1}^{N} \frac{1}{\sqrt{2\pi \sigma^2} } \exp\left( \frac{-(Y_{i} - \beta_{0} - \beta_{1}X_{i1} )^2}{2 \sigma^2} \right) \nonumber \\} \pause
\invisible<1-3>{& = & \frac{1}{\left( 2 \pi \sigma^2 \right)^{N/2}} \exp\left( - \sum_{i=1}^{N} \frac{ (Y_{i} - \beta_{0} - \beta_{1}X_{i1} )^2}{2 \sigma^2} \right) \nonumber }
\end{eqnarray}

\end{frame}


\begin{frame}
\frametitle{Bivariate Regression Model via Maximum Likelihood}

\begin{eqnarray}
L(\boldsymbol{\beta}, \sigma^2 | \boldsymbol{Y}, \boldsymbol{X}) & = & \frac{1}{\left( 2 \pi \sigma^2 \right)^{N/2}} \exp\left( - \sum_{i=1}^{N} \frac{ (Y_{i} - \beta_{0} - \beta_{1}X_{i1} )^2}{2 \sigma^2} \right) \nonumber \\ \pause
\invisible<1>{\log L(\boldsymbol{\beta}, \sigma^2 | \boldsymbol{Y}, \boldsymbol{X}) & = &  - \underbrace{\frac{N}{2} \log (2 \pi )}_{\text{constant}} - \frac{N}{2} \log (\sigma^2) - \sum_{i=1}^{N} \frac{ (Y_{i} - \beta_{0} - \beta_{1}X_{i1} )^2}{2 \sigma^2} \nonumber \\} \pause
\invisible<1-2>{l(\boldsymbol{\beta}, \sigma^2 | \boldsymbol{Y}, \boldsymbol{X}) & = & - \frac{N}{2} \log (\sigma^2) - \sum_{i=1}^{N} \frac{ (Y_{i} - \beta_{0} - \beta_{1}X_{i1} )^2}{2 \sigma^2} + c \nonumber }\pause
\end{eqnarray}

\invisible<1-3>{Optimize with respect to:
\begin{itemize}
\item[] $\boldsymbol{\beta}$
\item[] $\sigma^2$
\end{itemize}
} \pause
\end{frame}


\begin{frame}
\frametitle{Bivariate Regression Model via Maximum Likelihood}

\begin{eqnarray}
\frac{\partial l(\boldsymbol{\beta}, \sigma^2 | \boldsymbol{Y}, \boldsymbol{X})}{\partial \beta_{0}} & = &   \sum_{i=1}^{N} \frac{Y_{i} - \beta_{0} - \beta_{1} X_{i1} }{ \sigma^2} \nonumber \\
\frac{\partial l(\boldsymbol{\beta}, \sigma^2 | \boldsymbol{Y}, \boldsymbol{X})}{\partial \beta_{1}} & = &
   \sum_{i=1}^{N} \frac{Y_{i} - \beta_{0} - \beta_{1} X_{i1} }{ \sigma^2} X_{i1}  \nonumber \\
 \frac{\partial l(\boldsymbol{\beta}, \sigma^2 | \boldsymbol{Y}, \boldsymbol{X})}{\partial \sigma^2} & = & -\frac{N}{2 \sigma^2}  +    \sum_{i=1}^{N}  \frac{\left(Y_{i} - \beta_{0} - \beta_{1} X_{i1} \right)^2 }{ 2(\sigma^2) ^2 } \nonumber
\end{eqnarray}

\pause

\invisible<1>{Score function: $ s_{i}(\boldsymbol{\beta}, \sigma^2 ) = \nabla l(\boldsymbol{\beta}, \sigma^2 | Y_{i}, X_{i} ) = $
\begin{eqnarray}
\left(\frac{Y_{i} - \beta_{0} - \beta_{1} X_{i1} }{ \sigma^2},\frac{Y_{i} - \beta_{0} - \beta_{1} X_{i1} }{ \sigma^2}X_{i1}  , -\frac{1}{2 \sigma^2}  +      \frac{\left(Y_{i} - \beta_{0} - \beta_{1} X_{i1} \right)^2 }{ 2(\sigma^2) ^2 }  \right)  && \nonumber
\end{eqnarray}
} \pause 

\invisible<1-2>{$s_{N}(\boldsymbol{\beta}, \sigma^2) = \sum_{i=1}^{N} s_{i}(\boldsymbol{\beta}, \sigma^2 )$.  }

\end{frame}

\begin{frame}
\frametitle{Bivariate Regression Model via Maximum Likelihood}

\only<1>{\begin{eqnarray}
  0 & = & \sum_{i=1}^{N} \frac{Y_{i} - \beta_{0}^{*}  - \beta_{1}^{*} X_{i1} }{ \sigma^{2,*} } \nonumber \\
  0 & = &  \sum_{i=1}^{N} \frac{Y_{i} - \beta_{0}^{*} - \beta_{1}^{*} X_{i1} }{ \sigma^{2,*}}X_{i1} \nonumber \\
  0 &= & -\frac{N}{2 \sigma^{2,*}}  +    \sum_{i=1}^{N}  \frac{\left(Y_{i} - \beta_{0}^{*} - \beta_{1}^{*} X_{i1} \right)^2 }{ 2(\sigma^{2,*}) ^2 }\nonumber
\end{eqnarray}}


\only<2>{

	\huge
Algebra

}


\only<3>{

\begin{eqnarray}
\beta_{1}^{*} & = & \frac{\sum_{i=1}^{N} (X_{i} - \bar{X} )(Y_{i} - \bar{Y} ) }{\sum_{i=1}^{N} (X_{i}-  \bar{X})^2   } \nonumber  \\
\beta_{0}^{*} & = & \bar{Y} - \beta_{1}^{*} \bar{X} \nonumber \\
\sigma^{2,*} & = & \frac{1}{N} \sum_{i=1}^{N} (Y_{i} - \beta_{0}^{*} - \beta_1^{*}X_{i} )^2 \nonumber
\end{eqnarray}

}

\end{frame}


\begin{frame}
\frametitle{Hessian Matrix}

Define the Hessian for observation $i$ as 

\begin{eqnarray}
H_{i}(\boldsymbol{\beta}, \sigma^2) & = & \begin{pmatrix} 
\frac{\partial l(\boldsymbol{\beta}, \sigma^2|X_{i}, Y_{i})}{\partial \beta_{0}\beta_{0}} 
& \frac{\partial l(\boldsymbol{\beta}, \sigma^2|X_{i}, Y_{i})}{\partial \beta_{0}\beta_{1}} & \hdots & \frac{\partial l(\boldsymbol{\beta}, \sigma^2|X_{i}, Y_{i})}{\partial \beta_{0}\sigma^2} \\
\vdots & \vdots & \ddots & \vdots \\
\frac{\partial l(\boldsymbol{\beta}, \sigma^2|X_{i}, Y_{i})}{\partial \beta_{0}\sigma^2} & \frac{\partial l(\boldsymbol{\beta}, \sigma^2|X_{i}, Y_{i})}{\partial \beta_{1}\sigma^2} & \hdots & \frac{\partial l(\boldsymbol{\beta}, \sigma^2|X_{i}, Y_{i})}{\partial \sigma^2\sigma^2} 
\end{pmatrix} \nonumber 
\end{eqnarray}




\end{frame}



\begin{frame}
\frametitle{Bivariate Regression and Maximum Likelihood}

Fisher Information matrix
\begin{eqnarray}
I(\boldsymbol{\beta}, \sigma^2) & = & E[ s_{N}( \boldsymbol{\beta}, \sigma^2 ) s_{N}(\boldsymbol{\beta}, \sigma^2 )^{'}] \nonumber \\
I(\boldsymbol{\beta}, \sigma^2) & = & -E[ H_{i} (\boldsymbol{\beta}, \sigma^2)]  \text{ (under regularity conditions) }\nonumber \\
I_{ij}(\boldsymbol{\beta}, \sigma^2) & = & -E[\frac{\partial^2 l(\boldsymbol{\beta}, \sigma^2| Y_{i} , \boldsymbol{X}_{i})}{\partial \beta_{i} \beta_{j} }| \boldsymbol{\beta}, \sigma^2] \nonumber \\
I_{ij}(\boldsymbol{\beta}, \sigma^2) & = & -\int \frac{\partial^2 l(\boldsymbol{\beta}, \sigma^2| Y_{i}, \boldsymbol{X}_{i})}{\partial \beta_{i} \beta_{j} } f(Y_{i}|\boldsymbol{\beta}, \sigma^2, \boldsymbol{X} )d\boldsymbol{Y} \nonumber 
\end{eqnarray}

Observed Information Matrix

\begin{eqnarray}
I_{N}(\boldsymbol{\beta}, \sigma^2) & = & -\sum_{i=1}^{N} H_{i}(\boldsymbol{\beta}, \sigma^2) = -\sum_{i=1}^{N} \nabla^2 \log l(\boldsymbol{\beta}, \sigma^2|\boldsymbol{X}_{i}, Y_{i}) \nonumber 
\end{eqnarray}


\end{frame}


\begin{frame}
\frametitle{Three Ways to Estimate Variance}

\begin{eqnarray}
\widehat{\text{Var}(\boldsymbol{\beta}, \sigma^2)_{OS}} & = & \left[\sum_{i=1}^{N}  s_{i}(\boldsymbol{\beta}, \sigma^2 ) s_{i}( \boldsymbol{\beta}, \sigma^2 )^{'}\right]^{-1} (\text{BHHH estimator} ) \nonumber \\
\widehat{\text{Var}(\boldsymbol{\beta}, \sigma^2)_{OH}} & = & - \left[\sum_{i=1}^{N} H_{i}(\boldsymbol{\beta}, \sigma^2)\right]^{-1}  \nonumber \\
\widehat{\text{Var}(\boldsymbol{\beta}, \sigma^2)_{EH}} & = & - \left[\sum_{i=1}^{N} E[H_{i}(\boldsymbol{\beta}, \sigma^2) ]\right]^{-1}\nonumber
\end{eqnarray}



\end{frame}








\begin{frame}
\frametitle{Bivariate Regression Model via Maximum Likelihood}


\begin{eqnarray}
\frac{\partial^2 l(\boldsymbol{\beta}, \sigma^2 | \boldsymbol{Y}, \boldsymbol{X})}{\partial \beta_{0} \partial \beta_{0} } & = & - \sum_{i=1}^{N} \frac{1}{\sigma^{2} } \nonumber \\
\frac{\partial^2 l(\boldsymbol{\beta}, \sigma^2 | \boldsymbol{Y}, \boldsymbol{X})}{\partial \beta_{1} \partial \beta_{1} } &= & - \sum_{i=1}^{N} \frac{X_{i}^{2}}{\sigma^2} \nonumber \\
\frac{\partial^2 l(\boldsymbol{\beta}, \sigma^2 | \boldsymbol{Y}, \boldsymbol{X})}{\partial \sigma^2 \partial \sigma^2 } & = &  \frac{ N}{2 (\sigma^{2})^{2} } - \sum_{i=1}^{N} \frac{(Y_i - \beta_{0} - \beta_{1} X_{i1})^2 }{ (\sigma^2)^4 } \nonumber \\
\frac{\partial^2 l(\boldsymbol{\beta}, \sigma^2 | \boldsymbol{Y}, \boldsymbol{X})}{\partial \beta_{0} \partial \beta_{1} } & = & - \sum_{i=1}^{N} \frac{X_{i}}{\sigma^2}\nonumber \\
\frac{\partial^2 l(\boldsymbol{\beta}, \sigma^2 | \boldsymbol{Y}, \boldsymbol{X})}{\partial \sigma^2 \partial \beta_{1} } & = & - \sum_{i=1}^{N}\frac{(Y_{i} - \beta_{0} - \beta_{1} X_{i1})X_{i} }{(\sigma^2)^{2}} \nonumber
\end{eqnarray}


\end{frame}


\begin{frame}
\frametitle{Bivariate Regression Model via Maximum Likelihood}

\begin{eqnarray}
-E\left[\frac{\partial^2 l(\boldsymbol{\beta}, \sigma^2 | \boldsymbol{Y}, \boldsymbol{X})}{\partial \beta_{0} \partial \beta_{0} }\right ]  & = &  -E\left[- \frac{1}{\sigma^2} \right] \nonumber \\ \pause
\invisible<1>{& = & - \sum_{i=1}^{N} \int_{-\infty}^{\infty} -\frac{1}{\sigma^{2} } f(Y_{i} | \boldsymbol{\beta}, \sigma^2, X_{i}) dY_{i} \nonumber \\} \pause
\invisible<1-2>{& = & \sum_{i=1}^{N} \frac{1}{\sigma^2}\int_{-\infty}^{\infty} f(Y_{i} | \boldsymbol{\beta}, \sigma^2, X_{i}) dY_{i}\nonumber \\} \pause
\invisible<1-3>{& = & \sum_{i=1}^{N} \frac{1}{\sigma^2} \times 1  = \frac{N}{\sigma^2} \nonumber }
\end{eqnarray}

\end{frame}


\begin{frame}
\frametitle{Bivariate Regression Model via Maximum Likelihood}

\begin{eqnarray}
-E[\frac{\partial^2 l(\boldsymbol{\beta}, \sigma^2 | \boldsymbol{Y}, \boldsymbol{X})}{\partial \beta_{1} \partial \beta_{1} }]  &= &- E[ - \sum_{i=1}^{N} \frac{X_{i}^{2}}{\sigma^2}] \nonumber \\
					   & = & \sum_{i=1}^{N} \frac{X_{i}^2}{\sigma^2} \nonumber  \pause \\
\invisible<1>{-E[\frac{\partial^2 l(\boldsymbol{\beta}, \sigma^2 | \boldsymbol{Y}, \boldsymbol{X})}{\partial \beta_{0} \partial \beta_{1} }] & = & -E[- \sum_{i=1}^{N} \frac{X_{i}}{\sigma^2}]  \nonumber \\
					 & = & \sum_{i=1}^{N} \frac{X_{i}}{\sigma^2} \nonumber }
\end{eqnarray}


\end{frame}


\begin{frame}

\begin{eqnarray}
-E[\frac{\partial^2 l(\boldsymbol{\beta}, \sigma^2 | \boldsymbol{Y}, \boldsymbol{X})}{\partial \sigma^2 \partial \sigma^2 }] & = &  E[\frac{ N}{2 (\sigma^{2})^{2} } - \sum_{i=1}^{N} \frac{(Y_i - \beta_{0} - \beta_{1} X_{i1})^2 }{ (\sigma^2)^4 }] \nonumber  \pause \\
					\invisible<1>{& = & - \frac{N}{2 (\sigma^2)^2} +  \sum_{i=1}^{N} E\left[ \frac{(Y_i - \beta_{0} - \beta_{1} X_{i1})^2 }{ (\sigma^2)^4 }\right] \nonumber\\} \pause
					\invisible<1-2>{& = & - \frac{N}{2 (\sigma^2)^2} + \sum_{i=1}^{N} \frac{\sigma^2 }{ (\sigma^2)^4 } \nonumber \\} \pause
					\invisible<1-3>{& = & - \frac{N}{2 (\sigma^2)^2} + \frac{2 N }{ 2 (\sigma^2)^2 } \nonumber \\} \pause
					 \invisible<1-4>{& = & \frac{N}{ 2 (\sigma^2)^2 } \nonumber }
\end{eqnarray}

\end{frame}


\begin{frame}

\begin{eqnarray}
-E\left[\frac{\partial^2 l(\boldsymbol{\beta}, \sigma^2 | \boldsymbol{Y}, \boldsymbol{X})}{\partial \sigma^2 \partial \beta_{1} } \right] & = & -E\left[- \sum_{i=1}^{N}\frac{(Y_{i} - \beta_{0} - \beta_{1} X_{i1})X_{i} }{(\sigma^2)^{2}}\right]  \nonumber \pause \\
\invisible<1>{& = & \frac{X_{i} }{(\sigma^2)^2} \left(\sum_{i=1}^{N} E[Y_{i}] - \beta_{0} - \beta_{1} X_{i1}  \right) \nonumber \\} \pause
\invisible<1-2>{& = & \frac{X_{i} }{(\sigma^2)^2}\left(\sum_{i=1}^{N} \beta_{0} + \beta_{1} X_{i1}  - \beta_{0} - \beta_{1} X_{i1}  \right) = 0  \nonumber }
\end{eqnarray}



\end{frame}





\begin{frame}
\frametitle{Bivariate Regression Model via Maximum Likelihood}

\begin{eqnarray}
I(\boldsymbol{\beta}, \sigma^{2})_{N} & = & \begin{pmatrix}
\frac{N}{\sigma^2} &  \sum_{i=1}^{N} \frac{X_{i}}{\sigma^2} & 0 \\
 \sum_{i=1}^{N} \frac{X_{i}}{\sigma^2} & \sum_{i=1}^{N} \frac{X_{i}^2}{\sigma^2} & 0 \\
0 &   0 &  \frac{ N}{2 (\sigma^{2})^{2} } \\
\end{pmatrix} \nonumber \\
& = & \begin{pmatrix}
\frac{1}{\sigma^2}\boldsymbol{X}^{'}\boldsymbol{X} & \boldsymbol{0} \\
\boldsymbol{0} & \frac{ N}{2 (\sigma^{2})^{2} } \\
\end{pmatrix} \nonumber
\end{eqnarray}
Inverting yields:
\begin{eqnarray}
\widehat{\text{Var}(\boldsymbol{\beta}, \sigma^{2})} & = & \begin{pmatrix}
\sigma^2 \left(\boldsymbol{X}^{'}\boldsymbol{X}\right)^{-1} & 0 \\
0 & \frac{2 (\sigma^2)^2}{N} \\
\end{pmatrix} \nonumber
\end{eqnarray}

\end{frame}


\begin{frame}
\frametitle{Bivariate Regression with Maximum Likelihood: Numerical Simulation}


\only<1>{
\scalebox{0.5}{\includegraphics{SimDists_1.pdf}}	}
\only<2>{\scalebox{0.5}{\includegraphics{SimDists_2.pdf}}	}
\only<3>{\scalebox{0.5}{\includegraphics{SimDists_3.pdf}}	}
\only<4>{\scalebox{0.5}{\includegraphics{SimDists_4.pdf}}	}
\only<5>{\scalebox{0.5}{\includegraphics{SimDists_5.pdf}}	}

\end{frame}

\begin{frame}
\frametitle{Bivariate Regression via Maximum Likelihood}

\pause
\begin{eqnarray}
\invisible<1>{\text{Incumbent}_{i} & = & \beta^{*}_{0}  + \beta^{*}_{1} \text{Approval}_{i} + \epsilon_{i} \nonumber } \pause \\
\invisible<1-2>{\text{Incumbent}_{i} & = & \underbrace{50.76 + 0.16\times \text{Approval}_{i}}_{\widehat{\text{Incumbent}_{i}}}  + \epsilon_{i} \nonumber } \pause
\end{eqnarray}

\begin{center}
\scalebox{0.31}{\includegraphics{Bivariate2.pdf}}
\end{center}


\end{frame}



\begin{frame}
\frametitle{Multivariate Regression via Maximum Likelihood}

Define:
\begin{eqnarray}
\boldsymbol{X}_{i} & = & (1, X_{i1}, X_{i2}, X_{i3}) \nonumber \\
\boldsymbol{\beta} & = & (\beta_{0}, \beta_{1}, \beta_{2}) \nonumber
\end{eqnarray}

Then our maximum likelihood estimators are:
\begin{eqnarray}
\boldsymbol{\beta}^{*} & = & \left(\boldsymbol{X}^{'}\boldsymbol{X}   \right)^{-1} \boldsymbol{X}^{'}\boldsymbol{Y} \nonumber \\
\sigma^{2, *} & = & \frac{1}{N} (Y - \boldsymbol{X}\boldsymbol{\beta})^{'}(Y - \boldsymbol{X}\boldsymbol{\beta}) \nonumber \\
 & = & \frac{1}{N} \sum_{i=1}^{N} (Y_{i} - \beta_{0} - \beta_{1} X_{i1} - \beta_{2}X_{i2} - \beta_{3}X_{i3})^2\nonumber
\end{eqnarray}



\end{frame}

\begin{frame}


Then,

 \begin{eqnarray}
 \boldsymbol{\beta}^{*},\sigma^{2, *} & \rightarrow^{D} & \text{MVN}((\boldsymbol{\beta}^{*}, \sigma^{2,*}), I(\boldsymbol{\beta}^{*}, \sigma^{2,*})^{-1} ) \nonumber
 \end{eqnarray}

Where

\begin{eqnarray}
\widehat{\text{Var}(\boldsymbol{\beta}^{*}, \sigma^{2,*}) )} & = & \begin{pmatrix} \sigma^2 (\boldsymbol{X}^{'}\boldsymbol{X})^{-1} & 0 \\
0 & \frac{2 (\sigma^2)^2}{N} \\
\end{pmatrix} \nonumber
\end{eqnarray}

\end{frame}


\begin{frame}
\frametitle{Analytic vs Computational}

Analytic optimization:
\begin{itemize}
\item[-] Calculus + pencil and paper
\item[-] Search for closed form solutions
\item[-] Big guarantees, but solutions can be hard (impossible) to obtain
\end{itemize}

Computational Optimization:
\begin{itemize}
\item[-] Iterative procedure
\item[-] Search for \alert{approximate} solution
\item[-] Fewer guarantees, but you'll obtain solutions
\end{itemize}

Overall strategy:\\
One-parameter Newton Raphson$\leadsto$ Multi-parameter Newton Raphson $\leadsto$ BFGS (Wednesday)

\end{frame}


\begin{frame}
\frametitle{Newton-Raphson Method}

Iterative procedure to find a \alert{root} \\ \pause
\invisible<1>{Often solving for $x$ when $f(x) = 0$ is hard$\leadsto$ complicated function} \pause \\
\invisible<1-2>{Solving for $x$ when $f(x)$ is linear$\leadsto$ easy} \pause \\
\invisible<1-3>{Approximate with \alert{tangent line}, iteratively update}


%%put an example here that shows how this might work

%\scalebox{0.5}{\includegraphics{TangentExample.pdf}}

%use tangent line to find root
\end{frame}



\begin{frame}
\frametitle{Tangent Line}


\only<1>{\scalebox{0.6}{\includegraphics{TanLine26.pdf}}}
\only<2>{\scalebox{0.6}{\includegraphics{TanLine1.pdf}}}
\only<3>{\scalebox{0.6}{\includegraphics{TanLine2.pdf}}}
\only<4>{\scalebox{0.6}{\includegraphics{TanLine3.pdf}}}
\only<5>{\scalebox{0.6}{\includegraphics{TanLine4.pdf}}}
\only<6>{\scalebox{0.6}{\includegraphics{TanLine5.pdf}}}
\only<7>{\scalebox{0.6}{\includegraphics{TanLine6.pdf}}}
\only<8>{\scalebox{0.6}{\includegraphics{TanLine7.pdf}}}
\only<9>{\scalebox{0.6}{\includegraphics{TanLine8.pdf}}}
\only<10>{\scalebox{0.6}{\includegraphics{TanLine9.pdf}}}
\only<11>{\scalebox{0.6}{\includegraphics{TanLine10.pdf}}}
\only<12>{\scalebox{0.6}{\includegraphics{TanLine11.pdf}}}
\only<13>{\scalebox{0.6}{\includegraphics{TanLine12.pdf}}}
\only<14>{\scalebox{0.6}{\includegraphics{TanLine13.pdf}}}
\only<15>{\scalebox{0.6}{\includegraphics{TanLine14.pdf}}}
\only<16>{\scalebox{0.6}{\includegraphics{TanLine15.pdf}}}
\only<17>{\scalebox{0.6}{\includegraphics{TanLine16.pdf}}}
\only<18>{\scalebox{0.6}{\includegraphics{TanLine17.pdf}}}
\only<19>{\scalebox{0.6}{\includegraphics{TanLine18.pdf}}}
\only<20>{\scalebox{0.6}{\includegraphics{TanLine19.pdf}}}
\only<21>{\scalebox{0.6}{\includegraphics{TanLine20.pdf}}}
\only<22>{\scalebox{0.6}{\includegraphics{TanLine21.pdf}}}
\only<23>{\scalebox{0.6}{\includegraphics{TanLine22.pdf}}}
\only<24>{\scalebox{0.6}{\includegraphics{TanLine23.pdf}}}
\only<25>{\scalebox{0.6}{\includegraphics{TanLine24.pdf}}}
\only<26>{\scalebox{0.6}{\includegraphics{TanLine25.pdf}}}


\end{frame}



\begin{frame}
\frametitle{Tangent Line}

Formula for Tangent line at $x_{0}$:


\begin{eqnarray}
\invisible<1>{g(x) =  \alert<4>{f^{'}(x_{0})} \alert<5>{(x - x_{0} )} + \alert<3>{f(x_{0} )} \nonumber }
\end{eqnarray}


\invisible<1-5>{We'll use formula for tangent line to generate updates}


\pause \pause \pause \pause


\end{frame}






\begin{frame}
\frametitle{Newton-Raphson Method}

Suppose we have some initial guess $x_{0}$.  We're going to approximate $f^{'}(x)$ with the tangent line to generate a new guess


 \pause

\begin{eqnarray}
\invisible<1>{g(x) & = & f^{''}(x_{0})(x - x_{0} ) + f^{'}(x_{0} ) \nonumber \\}
\invisible<1-2>{0 & = & f^{''}(x_{0}) (x_{1} - x_{0}) + f^{'}(x_{0} ) \nonumber \\} %because at the root the function has to equal zero, so we're moving closer to it.
\invisible<1-3>{x_{1} & = &  x_{0} - \frac{f^{'}(x_{0}) }{f^{''}(x_{0})} \nonumber }
\end{eqnarray}

\invisible<1-4>{Perform iteratively until change in $|x_{t+1} - x_{t}|<$ threshold }

\end{frame}



\begin{frame}
\frametitle{Example Function }

$f(x) = x^3 + 2x^2 - 1$
find $x$ that maximizes $f(x) $ with $x \in [-3, 0]$


\scalebox{0.5}{\includegraphics{NRExample.pdf}}



\end{frame}



\begin{frame}

\begin{eqnarray}
f^{'}(x) & = & 3 x^2 + 4 x \nonumber \\
f^{''}(x) & = & 6x + 4 \nonumber
\end{eqnarray}

Suppose we have guess $x_{t}$ then the next step is:
\begin{eqnarray}
x_{t+1} & = & x_t - \frac{3 x_{t}^2 + 4 x_{t}}{6 x_{t} + 4} \nonumber
\end{eqnarray}


\end{frame}


\begin{frame}
\frametitle<7->{$x^{*} = -1.3333$}

\only<1>{\scalebox{0.5}{\includegraphics{NRExamp1.pdf}}}
\only<2>{\scalebox{0.5}{\includegraphics{NRExamp2.pdf}}}
\only<3>{\scalebox{0.5}{\includegraphics{NRExamp3.pdf}}}
\only<4>{\scalebox{0.5}{\includegraphics{NRExamp4.pdf}}}
\only<5>{\scalebox{0.5}{\includegraphics{NRExamp5.pdf}}}
\only<6>{\scalebox{0.5}{\includegraphics{NRExamp6.pdf}}}
\only<7>{\scalebox{0.5}{\includegraphics{NRExamp7.pdf}}}


\end{frame}


\begin{frame}
\frametitle{What is Happening with the Roots}

\only<1>{\scalebox{0.5}{\includegraphics{NewtRaphDeriv20.pdf}}}
\only<2>{\scalebox{0.5}{\includegraphics{NewtRaphDeriv1.pdf}}}
\only<3>{\scalebox{0.5}{\includegraphics{NewtRaphDeriv2.pdf}}}
\only<4>{\scalebox{0.5}{\includegraphics{NewtRaphDeriv3.pdf}}}
\only<5>{\scalebox{0.5}{\includegraphics{NewtRaphDeriv4.pdf}}}
\only<6>{\scalebox{0.5}{\includegraphics{NewtRaphDeriv5.pdf}}}
\only<7>{\scalebox{0.5}{\includegraphics{NewtRaphDeriv6.pdf}}}
\only<8>{\scalebox{0.5}{\includegraphics{NewtRaphDeriv7.pdf}}}
\only<9>{\scalebox{0.5}{\includegraphics{NewtRaphDeriv8.pdf}}}
\only<10>{\scalebox{0.5}{\includegraphics{NewtRaphDeriv9.pdf}}}
\only<11>{\scalebox{0.5}{\includegraphics{NewtRaphDeriv10.pdf}}}
\only<12>{\scalebox{0.5}{\includegraphics{NewtRaphDeriv11.pdf}}}
\only<13>{\scalebox{0.5}{\includegraphics{NewtRaphDeriv12.pdf}}}





\end{frame}



\begin{frame}


\only<1>{\scalebox{0.5}{\includegraphics{NRExamp8.pdf}}}
\only<2>{\scalebox{0.5}{\includegraphics{NRExamp9.pdf}}}
\only<3>{\scalebox{0.5}{\includegraphics{NRExamp10.pdf}}}
\only<4>{\scalebox{0.5}{\includegraphics{NRExamp11.pdf}}}
\only<5>{\scalebox{0.5}{\includegraphics{NRExamp12.pdf}}}
\only<6>{\scalebox{0.5}{\includegraphics{NRExamp13.pdf}}}
\only<7>{\scalebox{0.5}{\includegraphics{NRExamp14.pdf}}}
\only<8>{\scalebox{0.5}{\includegraphics{NRExamp15.pdf}}}
\only<9>{\scalebox{0.5}{\includegraphics{NRExamp16.pdf}}}
\only<10>{\scalebox{0.5}{\includegraphics{NRExamp17.pdf}}}
\only<11>{\scalebox{0.5}{\includegraphics{NRExamp18.pdf}}}
\only<12>{\scalebox{0.5}{\includegraphics{NRExamp19.pdf}}}
\end{frame}

\begin{frame}
\frametitle{Multivariate Newton Raphson}

Suppose $f:\Re^{n} \rightarrow \Re$.  Suppose we have guess $\boldsymbol{x}_{t}$. \pause \invisible<1>{  Then our update is:} \pause
\begin{eqnarray}
\invisible<1-2>{\boldsymbol{x}_{t+1} & = & \boldsymbol{x}_{t} - \underbrace{\textbf{H}(f)(\boldsymbol{x}_{t})^{-1}}_{\text{Hessian}} \underbrace{\nabla f(\boldsymbol{x}_{t}) }_{\text{gradient}}}\nonumber \pause
\end{eqnarray}

\invisible<1-3>{Derivation (intuition):}\pause \invisible<1-4>{ Approximate function with \alert{tangent plane}. } \pause \invisible<1-5>{ Find value of $x_{t+1}$ that makes the plane equal to zero.  Update again.  }


\end{frame}


\begin{frame}
\frametitle{Multivariate Newton Raphson: Multivariate Linear Regression Model}

Suppose that we have guess:
$\left(\boldsymbol{\beta}_{t}, \sigma^{2}_{t}\right)$


\begin{eqnarray}
(\boldsymbol{\beta}_{t+1}, \sigma^{2}_{t+1} & = & \left(\boldsymbol{\beta}_{t}, \sigma^{2}_{t} \right) \underbrace{- \textbf{H}(f)(\left(\boldsymbol{\beta}_{t}, \sigma^{2}_{t} \right))^{-1}}_{\text{Information}} \underbrace{\nabla f(\left(\boldsymbol{\beta}_{t}, \sigma^{2}_{t} \right)}_{\text{Score}}\nonumber \\
\left(\boldsymbol{\beta}_{t+1}, \sigma^{2}_{t+1} \right) & = & \left(\boldsymbol{\beta}_{t}, \sigma^{2}_{t} \right) +  I_{N}(\boldsymbol{\beta}_{t}, \sigma^{2}_{t})^{-1} \left(\sum_{i=1}^{N} s(Y_{i}| X_{i}, \boldsymbol{\beta}_{t}, \sigma^{2}_{t})  \right) \nonumber
\end{eqnarray}

So we have:
\begin{tiny}
\begin{eqnarray}
\left(\boldsymbol{\beta}_{t+1}, \sigma^{2}_{t+1} \right)  =  \left(\boldsymbol{\beta}_{t}, \sigma^{2}_{t} \right)  && \nonumber \\
 + \begin{pmatrix} \sigma^2 \left(\boldsymbol{X}^{'}\boldsymbol{X}\right)^{-1} & \boldsymbol{0} \\
\boldsymbol{0} & \frac{2 (\sigma^2)^2}{N} \\
\end{pmatrix}  \left(\sum_{i=1}^{N} \left( \frac{(\boldsymbol{Y} - \boldsymbol{X}\boldsymbol{\beta}_{t})^{'} \boldsymbol{X} }{\sigma^2}  , -\frac{1}{2 \sigma^2}  +      \frac{1}{2\sigma^4} (\boldsymbol{Y} - \boldsymbol{X}\boldsymbol{\beta} )^{'}(\boldsymbol{Y} - \boldsymbol{X} \boldsymbol{\beta})  \right)    \right)  && \nonumber
\end{eqnarray}
\end{tiny}



\end{frame}




\begin{frame}
\frametitle{Multivariate Newton Raphson}

\pause

\begin{itemize}
\invisible<1>{\item[-] Expensive to calculate (requires inverting Hessian)} \pause
\invisible<1-2>{\item[-] Very sensitive to starting points} \pause
\invisible<1-3>{\item[-] Ideally: method that exploits Newton-like structure, but is cheaper and more robust} \pause
\end{itemize}

\invisible<1-4>{{\tt BFGS}: \alert{Quasi-Newton} method\\} \pause
\invisible<1-5>{{\tt R code}}


\end{frame}



\begin{frame}

\begin{itemize}
\item[1)] Multivariate regression
\item[2)] \alert{Information}$\leadsto$ learning about uncertainty from model
\item[3)] Numerical optimization
\end{itemize}

Probit/Logit + BFGS next!!



\end{frame}





\end{document}
